\documentclass{ltjsarticle}

% 日本語設定
\usepackage{luatexja}
\usepackage{luatexja-fontspec}

% IPAex フォント
\setmainfont{IPAexGothic}
\setsansfont{IPAexGothic}
\setmonofont{IPAexGothic}

% LuaTeX-ja 拡張設定
\ltjsetparameter{jacharrange={-2}}

% Pandoc が生成するリスト密度調整用
\providecommand{\tightlist}{%
  \setlength{\itemsep}{0pt}\setlength{\parskip}{0pt}}

% 一般パッケージ
\usepackage{amsmath,amssymb}
\usepackage{graphicx}
\usepackage{url}

\usepackage{longtable,booktabs,array}
\newcounter{none}


% ページ設定
\usepackage[a4paper,margin=25mm]{geometry}

% 行間

% 目次

\begin{document}

% タイトル


\date{}


% 概要

% 目次

% 本文
lilcaSoundDriverFC 仕様書

\section{負荷}\label{ux8ca0ux8377}

\begin{verbatim}
 VBlink : 20 x 341 = 6820[cyc]
 OAM転送 : 512[cyc]

 サウンドドライバ: 820 + 55 x 2 [cyc]
 曲のデータサイズ: 159[byte](The Theme of lilca)
\end{verbatim}

\section{メタ情報}\label{ux30e1ux30bfux60c5ux5831}

\subsubsection{変換されたコードのコメントとして出力されます}\label{ux5909ux63dbux3055ux308cux305fux30b3ux30fcux30c9ux306eux30b3ux30e1ux30f3ux30c8ux3068ux3057ux3066ux51faux529bux3055ux308cux307eux3059}

\begin{itemize}
\tightlist
\item
  書式
\end{itemize}

\begin{verbatim}
 #<META_NAME> <string>
\end{verbatim}

\begin{itemize}
\tightlist
\item
  FC側で使う可能性のあるメタ情報
\end{itemize}

{\def\LTcaptype{none} % do not increment counter
\begin{longtable}[]{@{}ccl@{}}
\toprule\noalign{}
項目 & キーワード & 備考 \\
\midrule\noalign{}
\endhead
\bottomrule\noalign{}
\endlastfoot
曲名 & \#TITLE & \\
作曲者 & \#COMPOSER & \\
プログラマ & \#PROGRAMER & MMLやサウンドデータに起こした人 \\
ラベル & \#LABEL & プログラム上で使われるラベル名(*1) \\
\end{longtable}
}

(*1 mck mmlの仕様にない)

\subsubsection{いずれのメタ情報もキーワード直後の区切り文字以降の文字列を前後トリミングして値とする}\label{ux3044ux305aux308cux306eux30e1ux30bfux60c5ux5831ux3082ux30adux30fcux30efux30fcux30c9ux76f4ux5f8cux306eux533aux5207ux308aux6587ux5b57ux4ee5ux964dux306eux6587ux5b57ux5217ux3092ux524dux5f8cux30c8ux30eaux30dfux30f3ux30b0ux3057ux3066ux5024ux3068ux3059ux308b}

\begin{verbatim}
例:
 "#COMPOSER   lilca reload   "

 Meta Value = "lilca reload"
\end{verbatim}

\section{マクロ}\label{ux30deux30afux30ed}

\begin{itemize}
\tightlist
\item
  一覧
\end{itemize}

{\def\LTcaptype{none} % do not increment counter
\begin{longtable}[]{@{}
  >{\raggedright\arraybackslash}p{(\linewidth - 12\tabcolsep) * \real{0.1053}}
  >{\centering\arraybackslash}p{(\linewidth - 12\tabcolsep) * \real{0.1579}}
  >{\centering\arraybackslash}p{(\linewidth - 12\tabcolsep) * \real{0.1579}}
  >{\centering\arraybackslash}p{(\linewidth - 12\tabcolsep) * \real{0.1579}}
  >{\centering\arraybackslash}p{(\linewidth - 12\tabcolsep) * \real{0.1579}}
  >{\centering\arraybackslash}p{(\linewidth - 12\tabcolsep) * \real{0.1579}}
  >{\raggedright\arraybackslash}p{(\linewidth - 12\tabcolsep) * \real{0.1053}}@{}}
\toprule\noalign{}
\begin{minipage}[b]{\linewidth}\raggedright
マクロ名
\end{minipage} & \begin{minipage}[b]{\linewidth}\centering
定義
\end{minipage} & \begin{minipage}[b]{\linewidth}\centering
使用
\end{minipage} & \begin{minipage}[b]{\linewidth}\centering
ループ指定
\end{minipage} & \begin{minipage}[b]{\linewidth}\centering
終端子
\end{minipage} & \begin{minipage}[b]{\linewidth}\centering
有効チャンネル
\end{minipage} & \begin{minipage}[b]{\linewidth}\raggedright
備考
\end{minipage} \\
\midrule\noalign{}
\endhead
\bottomrule\noalign{}
\endlastfoot
ボリューム & @v {[}n{]} & @v {[}n{]} & あり(*2) & - & 12N (*1) &
フレーム単位でボリュームを変化させる \\
音色 & @{[}n{]} & @@{[}n{]} & あり(*2) & - & 12N (*1) &
フレーム単位で音色を変化させる \\
アルペジオ(*3) & @EN {[}n{]} & EN{[}n{]} & あり(*2) & ENOF & 12TN (*1) &
フレーム単位で音階を変化させる \\
ピッチ(*3) & @EP {[}n{]} & EP0 & あり(*2) & EPOF & 12TN (*1) &
フレーム単位で指定方向と量で音階を変化させる \\
ビブラート(*3) & @MP {[}n{]} & MP{[}n{]} & - & MPOF & 12TN (*1) &
指定された周期と振幅のSin波で音階を変化させる \\
DPCM割り当て(*3) & @DPCM {[}n{]} & 音階 & - & - & D (*1) &
n=0,1,2,3,4\ldots{}-\textgreater{} c,c+,d,d+,e\ldots{}に割り当て \\
\end{longtable}
}

*1 チャンネル名: 1=矩形波1, 2=矩形波2, T=三角波, N=ノイズ波, D=DPCM

*2 ``\textbar{}''以降をループする

*3 未対応

\section{マクロパラメータ値}\label{ux30deux30afux30edux30d1ux30e9ux30e1ux30fcux30bfux5024}

\subsection{マクロ値}\label{ux30deux30afux30edux5024}

{\def\LTcaptype{none} % do not increment counter
\begin{longtable}[]{@{}ccl@{}}
\toprule\noalign{}
ビット & 範囲 & 説明 \\
\midrule\noalign{}
\endhead
\bottomrule\noalign{}
\endlastfoot
1000-0000 & -128 & ループバックシンボル(次の1バイトが戻る値) \\
1000-00010111-1110 & -127 \textasciitilde{} 126 & マクロ値 \\
0111-1111 & 127 & 未使用 \\
\end{longtable}
}

\subsection{ボリューム, 音色, アルペジオ,
ピッチ共通}\label{ux30dcux30eaux30e5ux30fcux30e0-ux97f3ux8272-ux30a2ux30ebux30daux30b8ux30aa-ux30d4ux30c3ux30c1ux5171ux901a}

{\def\LTcaptype{none} % do not increment counter
\begin{longtable}[]{@{}
  >{\raggedright\arraybackslash}p{(\linewidth - 4\tabcolsep) * \real{0.3333}}
  >{\raggedright\arraybackslash}p{(\linewidth - 4\tabcolsep) * \real{0.3333}}
  >{\raggedright\arraybackslash}p{(\linewidth - 4\tabcolsep) * \real{0.3333}}@{}}
\toprule\noalign{}
\begin{minipage}[b]{\linewidth}\raggedright
値
\end{minipage} & \begin{minipage}[b]{\linewidth}\raggedright
範囲
\end{minipage} & \begin{minipage}[b]{\linewidth}\raggedright
説明
\end{minipage} \\
\midrule\noalign{}
\endhead
\bottomrule\noalign{}
\endlastfoot
ボリューム値 & 0\textasciitilde15 & \\
音色 & 0\textasciitilde3 & 矩形波=duty比ノイズ波=ノイズモード \\
相対音階 & -127\textasciitilde126 & cに対して1 は c+-1 は b \\
増減値 & -127\textasciitilde126 & 矩形波と三角波 -\textgreater{}
11bitタイマーノイズ波 -\textgreater{} 4bitピリオド \\
\end{longtable}
}

\begin{verbatim}
 @v0={ 15 8 4 2 0}  ; 1フレーム目 15 ... 5フレーム目 0 それ以降 0
 @v1={ 15 |8 4 }    ; 1フレーム目 15, 2フレーム目 8, 3フレーム目 4, それ以降 8と4を繰り返す
 @v2={|15 15 12 11 10 8 8 8 } ; '|' 以降の数値を繰り返す
 @ep0={2 |4 0}  ; 1フレーム目 +2, 2フレーム目 +6, 3フレーム目 +6, 以降 4,0増加させる (*)
 @ep1={-16}; 1フレーム目 変化なし, 2フレーム目 16減少, 3フレーム目 さらに16減少, 以降 16現状させる(*)
\end{verbatim}

(* 最大値を超えたら最大値に、最小値を下回ったら最小値にする)

\subsection{ビブラート}\label{ux30d3ux30d6ux30e9ux30fcux30c8}

{\def\LTcaptype{none} % do not increment counter
\begin{longtable}[]{@{}
  >{\raggedright\arraybackslash}p{(\linewidth - 4\tabcolsep) * \real{0.3333}}
  >{\raggedright\arraybackslash}p{(\linewidth - 4\tabcolsep) * \real{0.3333}}
  >{\raggedright\arraybackslash}p{(\linewidth - 4\tabcolsep) * \real{0.3333}}@{}}
\toprule\noalign{}
\begin{minipage}[b]{\linewidth}\raggedright
値
\end{minipage} & \begin{minipage}[b]{\linewidth}\raggedright
範囲
\end{minipage} & \begin{minipage}[b]{\linewidth}\raggedright
説明
\end{minipage} \\
\midrule\noalign{}
\endhead
\bottomrule\noalign{}
\endlastfoot
遅延 & 0\textasciitilde255 & Sin波を適用するまでのフレーム数 \\
速さ & 1\textasciitilde255 & Sin波の周期 \\
深さ & 0\textasciitilde255 & Sin波の振幅矩形波と三角波 -\textgreater{}
11bitタイマーノイズ波 -\textgreater{} 4bitピリオド \\
\end{longtable}
}

\begin{verbatim}
@mp={4 8 6} ; 4フレーム変化ないし,それ以降 周期8フレーム, 振幅6のSin波を適用する
\end{verbatim}

\subsection{DPCM}\label{dpcm}

{\def\LTcaptype{none} % do not increment counter
\begin{longtable}[]{@{}
  >{\raggedright\arraybackslash}p{(\linewidth - 4\tabcolsep) * \real{0.3333}}
  >{\raggedright\arraybackslash}p{(\linewidth - 4\tabcolsep) * \real{0.3333}}
  >{\raggedright\arraybackslash}p{(\linewidth - 4\tabcolsep) * \real{0.3333}}@{}}
\toprule\noalign{}
\begin{minipage}[b]{\linewidth}\raggedright
値
\end{minipage} & \begin{minipage}[b]{\linewidth}\raggedright
範囲
\end{minipage} & \begin{minipage}[b]{\linewidth}\raggedright
説明
\end{minipage} \\
\midrule\noalign{}
\endhead
\bottomrule\noalign{}
\endlastfoot
ファイルパス & 文字列 & 実行ディレクトリからの相対パス \\
パラメータ1 & 0\textasciitilde15 & ピッチ 15=ノーマル
(*)\(4010 b3-0のRate Index|
|パラメータ2|0~4081|再生するブロック数(16byte単位)<br>0または省略時=ファイルサイズ (*)|
|パラメータ3|0~15|出力レベル(波形のスタート地点) (*)<br>\)FF
推奨この値から波形が増減する \\
パラメータ4 & 0\textasciitilde2 & 再生モード (*)0=1ショット, 1=ループ \\
\end{longtable}
}

(* 後方省略可能だが、中間のパラメータは省略できない)

\begin{itemize}
\tightlist
\item
  (著者用のメモ)
\end{itemize}

\begin{verbatim}
 <Macro> ::= "@" <MacroName> <M_Number> "=" "{" <NumberList> "}"

 <NumberList> ::= <Number>+                -- ループなし
               |  <Number>+ "|" <Number>+  -- ループあり
               |  "|" <Number>+            -- 冒頭からループ
               |  <Delay> <Speed> <Depth>  -- ビブラートのとき

 <MacroName> ::= "v" | "@" | "en" | "ep" | "mp"

 <M_Number>  ::= 0-127
              |  0-63           -- ビブラートのとき

 <Number>    ::=    0 ~ 15      -- ボリューム
              |     0 ~ 3       -- 音色
              |  -127 ~ 126     -- アルペジオ
              |  -127 ~ 126     -- ピッチ

 <Delay> ::= 0 ~ 255
 <Speed> ::= 1 ~ 255
 <Depth> ::= 0 ~ 255
\end{verbatim}

\section{チャンネル}\label{ux30c1ux30e3ux30f3ux30cdux30eb}

{\def\LTcaptype{none} % do not increment counter
\begin{longtable}[]{@{}cc@{}}
\toprule\noalign{}
チャンネル名 & シンボル \\
\midrule\noalign{}
\endhead
\bottomrule\noalign{}
\endlastfoot
矩形波チャンネル1 & A \\
矩形波チャンネル2 & B \\
三角波チャンネル & C \\
ノイズチャンネル & D \\
DPCMチャンネル & E \\
\end{longtable}
}

\section{内部フォーマット(構文解析のみ)}\label{ux5185ux90e8ux30d5ux30a9ux30fcux30deux30c3ux30c8ux69cbux6587ux89e3ux6790ux306eux307f}

{\def\LTcaptype{none} % do not increment counter
\begin{longtable}[]{@{}
  >{\raggedright\arraybackslash}p{(\linewidth - 6\tabcolsep) * \real{0.2500}}
  >{\raggedright\arraybackslash}p{(\linewidth - 6\tabcolsep) * \real{0.2500}}
  >{\raggedright\arraybackslash}p{(\linewidth - 6\tabcolsep) * \real{0.2500}}
  >{\raggedright\arraybackslash}p{(\linewidth - 6\tabcolsep) * \real{0.2500}}@{}}
\toprule\noalign{}
\begin{minipage}[b]{\linewidth}\raggedright
コマンド値
\end{minipage} & \begin{minipage}[b]{\linewidth}\raggedright
意味
\end{minipage} & \begin{minipage}[b]{\linewidth}\raggedright
サイズ
\end{minipage} & \begin{minipage}[b]{\linewidth}\raggedright
備考
\end{minipage} \\
\midrule\noalign{}
\endhead
\bottomrule\noalign{}
\endlastfoot
\$0\_ nn ff & 音符 & nn = ノート番号 ff = フレーム数(*1) & \\
\$1m ff & 休符 & m = モード(0=音止める, other=音止めない) ff =
フレーム数(*1) & \\
\$2v & 音量 & v = 音量値(0-15) & \\
\$3t & 音色 & t = 音色: 矩形波(0-3) ノイズ(0-1) & \\
\$4\_ pp pp & ピッチ & pppp = ピッチ量(-32768\textasciitilde32767)(*3)
& \\
\$5o & オクターブ & o = オクターブ値(0-5 =\textgreater{} o2-o7) & \\
\$6\_ & オクターブUP & 引数なし & \\
\$7\_ & オクターブDOWN & 引数なし & \\
\$80(*4) & ボリュームマクロ & マクロ番号 & (*2) \\
\$81 & 音色マクロ & マクロ番号 & (*2) \\
\$82 & アルペジオマクロ開始 & マクロ番号 & (*2) \\
\$83 & ピッチマクロ & マクロ番号 & (*2) \\
\$84 & ビブラートマクロ開始 & マクロ番号 & (*2) \\
\$8C & アルペジオマクロ終了 & なし & \\
\$8D & ピッチマクロ終了 & なし & \\
\$8E & ビブラートマクロ終了 & なし & \\
\(A_        | リピート開始    | なし(リピートカウントに\)FFを設定) & &
& \\
\$Br & リピート終了 & r = リピート回数(0-15) & \\
\$FF & 終端子 & 引数なし & \\
\end{longtable}
}

\begin{verbatim}
廃止
| $80<br>(無くなるかも)| ボリュームマクロ | マクロ番号 | (*2) |
| $81        | 音色マクロ     | マクロ番号 | (*2) |
| $82        | アルペジオマクロ開始 | マクロ番号 | (*2) |
| $83        | ピッチマクロ    | マクロ番号 | (*2) |
| $84        | ビブラートマクロ開始 | マクロ番号 | (*2) |
| $92        | アルペジオマクロ終了 | なし | |
| $93        | ピッチマクロ終了 | なし | |
| $94        | ビブラートマクロ終了 | なし |
\end{verbatim}

*1 省略時は ``-1'' が入る

*2 バイナリ変換時に番号は振り直すのでmmlの番号とは一致しない

*3 ピッチ量は通常0

*4 \(D0~\)DFは

\section{音階}\label{ux97f3ux968e}

\begin{itemize}
\tightlist
\item
  音階n: C1=0 - A8=93
\end{itemize}

\begin{verbatim}
 a4の周波数 = 440
 周波数f: = 440 x 2 xx (n / 12)
\end{verbatim}

\begin{itemize}
\tightlist
\item
  算出式
\end{itemize}

\begin{verbatim}
 f = fn(note, octave)
   = 440 * 2 ** (( (note-9) + (octave-4) * 12 -  )/12)
タイマー値T: = -1 + 1789773 / 16 / f
\end{verbatim}

\section{長さ(音符, 休符)}\label{ux9577ux3055ux97f3ux7b26-ux4f11ux7b26}

\begin{itemize}
\tightlist
\item
  定義
\end{itemize}

\begin{verbatim}
 テンポ t (例 = 160)
 符長 len (例 = 32 = 32分音符)

 * 符長: 音符や休符の長さ
\end{verbatim}

\begin{itemize}
\tightlist
\item
  定数A
\end{itemize}

\begin{verbatim}
 A = 60fps x 60秒 x 4分音符 = 14400
\end{verbatim}

\begin{itemize}
\tightlist
\item
  符長のフレーム数 F
\end{itemize}

\begin{verbatim}
 F =  Int (A / t / len)

 * Int()で切り捨てられた少数点以下の値は累積され、
   "1"を超えると音符や休符を1フレーム長くして調整している
\end{verbatim}

\section{デフォルト値}\label{ux30c7ux30d5ux30a9ux30ebux30c8ux5024}

{\def\LTcaptype{none} % do not increment counter
\begin{longtable}[]{@{}lc@{}}
\toprule\noalign{}
項目 & デフォルト値 \\
\midrule\noalign{}
\endhead
\bottomrule\noalign{}
\endlastfoot
テンポ & 120 \\
長さ & 4 \\
ボリューム & 12 \\
オクターブ & 4 \\
\end{longtable}
}

\subsubsection{タイマー値の計算}\label{ux30bfux30a4ux30deux30fcux5024ux306eux8a08ux7b97}

f = clock / (16 *(timer + 1)) timer = clock / (16 x f) - 1

\#\#\#d 未整理メモf ノイズチャンネル @v1=\{12 15 14 14 13 13 \textbar{}
8 8 8 8 3 3 3 3 0 \} ; 音階 c - b 0-4 5 6 7 8 9 a b c d e f ;\$400E b3-0
= 0 1 2 3 4 5 6 7 8 9 a b 音階 c d e f g a b 周期 5 6 7 8 9 a b
0-4,c-fが使えない

\section{参考文献}\label{ux53c2ux8003ux6587ux732e}

\href{https://woolyss.com/chipmusic/chipmusic-mml/ppmck_guide.php}{Ultimate
PPMCK MML Reference}

\href{https://www.nesdev.org/mck_guide_v1.0.txt}{MCK/MML BEGINNERS
GUIDE}

% 参考文献

\end{document}
